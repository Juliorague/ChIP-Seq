% Options for packages loaded elsewhere
\PassOptionsToPackage{unicode}{hyperref}
\PassOptionsToPackage{hyphens}{url}
%
\documentclass[
]{article}
\usepackage{amsmath,amssymb}
\usepackage{iftex}
\ifPDFTeX
  \usepackage[T1]{fontenc}
  \usepackage[utf8]{inputenc}
  \usepackage{textcomp} % provide euro and other symbols
\else % if luatex or xetex
  \usepackage{unicode-math} % this also loads fontspec
  \defaultfontfeatures{Scale=MatchLowercase}
  \defaultfontfeatures[\rmfamily]{Ligatures=TeX,Scale=1}
\fi
\usepackage{lmodern}
\ifPDFTeX\else
  % xetex/luatex font selection
\fi
% Use upquote if available, for straight quotes in verbatim environments
\IfFileExists{upquote.sty}{\usepackage{upquote}}{}
\IfFileExists{microtype.sty}{% use microtype if available
  \usepackage[]{microtype}
  \UseMicrotypeSet[protrusion]{basicmath} % disable protrusion for tt fonts
}{}
\makeatletter
\@ifundefined{KOMAClassName}{% if non-KOMA class
  \IfFileExists{parskip.sty}{%
    \usepackage{parskip}
  }{% else
    \setlength{\parindent}{0pt}
    \setlength{\parskip}{6pt plus 2pt minus 1pt}}
}{% if KOMA class
  \KOMAoptions{parskip=half}}
\makeatother
\usepackage{xcolor}
\usepackage[margin=1in]{geometry}
\usepackage{color}
\usepackage{fancyvrb}
\newcommand{\VerbBar}{|}
\newcommand{\VERB}{\Verb[commandchars=\\\{\}]}
\DefineVerbatimEnvironment{Highlighting}{Verbatim}{commandchars=\\\{\}}
% Add ',fontsize=\small' for more characters per line
\usepackage{framed}
\definecolor{shadecolor}{RGB}{248,248,248}
\newenvironment{Shaded}{\begin{snugshade}}{\end{snugshade}}
\newcommand{\AlertTok}[1]{\textcolor[rgb]{0.94,0.16,0.16}{#1}}
\newcommand{\AnnotationTok}[1]{\textcolor[rgb]{0.56,0.35,0.01}{\textbf{\textit{#1}}}}
\newcommand{\AttributeTok}[1]{\textcolor[rgb]{0.13,0.29,0.53}{#1}}
\newcommand{\BaseNTok}[1]{\textcolor[rgb]{0.00,0.00,0.81}{#1}}
\newcommand{\BuiltInTok}[1]{#1}
\newcommand{\CharTok}[1]{\textcolor[rgb]{0.31,0.60,0.02}{#1}}
\newcommand{\CommentTok}[1]{\textcolor[rgb]{0.56,0.35,0.01}{\textit{#1}}}
\newcommand{\CommentVarTok}[1]{\textcolor[rgb]{0.56,0.35,0.01}{\textbf{\textit{#1}}}}
\newcommand{\ConstantTok}[1]{\textcolor[rgb]{0.56,0.35,0.01}{#1}}
\newcommand{\ControlFlowTok}[1]{\textcolor[rgb]{0.13,0.29,0.53}{\textbf{#1}}}
\newcommand{\DataTypeTok}[1]{\textcolor[rgb]{0.13,0.29,0.53}{#1}}
\newcommand{\DecValTok}[1]{\textcolor[rgb]{0.00,0.00,0.81}{#1}}
\newcommand{\DocumentationTok}[1]{\textcolor[rgb]{0.56,0.35,0.01}{\textbf{\textit{#1}}}}
\newcommand{\ErrorTok}[1]{\textcolor[rgb]{0.64,0.00,0.00}{\textbf{#1}}}
\newcommand{\ExtensionTok}[1]{#1}
\newcommand{\FloatTok}[1]{\textcolor[rgb]{0.00,0.00,0.81}{#1}}
\newcommand{\FunctionTok}[1]{\textcolor[rgb]{0.13,0.29,0.53}{\textbf{#1}}}
\newcommand{\ImportTok}[1]{#1}
\newcommand{\InformationTok}[1]{\textcolor[rgb]{0.56,0.35,0.01}{\textbf{\textit{#1}}}}
\newcommand{\KeywordTok}[1]{\textcolor[rgb]{0.13,0.29,0.53}{\textbf{#1}}}
\newcommand{\NormalTok}[1]{#1}
\newcommand{\OperatorTok}[1]{\textcolor[rgb]{0.81,0.36,0.00}{\textbf{#1}}}
\newcommand{\OtherTok}[1]{\textcolor[rgb]{0.56,0.35,0.01}{#1}}
\newcommand{\PreprocessorTok}[1]{\textcolor[rgb]{0.56,0.35,0.01}{\textit{#1}}}
\newcommand{\RegionMarkerTok}[1]{#1}
\newcommand{\SpecialCharTok}[1]{\textcolor[rgb]{0.81,0.36,0.00}{\textbf{#1}}}
\newcommand{\SpecialStringTok}[1]{\textcolor[rgb]{0.31,0.60,0.02}{#1}}
\newcommand{\StringTok}[1]{\textcolor[rgb]{0.31,0.60,0.02}{#1}}
\newcommand{\VariableTok}[1]{\textcolor[rgb]{0.00,0.00,0.00}{#1}}
\newcommand{\VerbatimStringTok}[1]{\textcolor[rgb]{0.31,0.60,0.02}{#1}}
\newcommand{\WarningTok}[1]{\textcolor[rgb]{0.56,0.35,0.01}{\textbf{\textit{#1}}}}
\usepackage{longtable,booktabs,array}
\usepackage{calc} % for calculating minipage widths
% Correct order of tables after \paragraph or \subparagraph
\usepackage{etoolbox}
\makeatletter
\patchcmd\longtable{\par}{\if@noskipsec\mbox{}\fi\par}{}{}
\makeatother
% Allow footnotes in longtable head/foot
\IfFileExists{footnotehyper.sty}{\usepackage{footnotehyper}}{\usepackage{footnote}}
\makesavenoteenv{longtable}
\usepackage{graphicx}
\makeatletter
\def\maxwidth{\ifdim\Gin@nat@width>\linewidth\linewidth\else\Gin@nat@width\fi}
\def\maxheight{\ifdim\Gin@nat@height>\textheight\textheight\else\Gin@nat@height\fi}
\makeatother
% Scale images if necessary, so that they will not overflow the page
% margins by default, and it is still possible to overwrite the defaults
% using explicit options in \includegraphics[width, height, ...]{}
\setkeys{Gin}{width=\maxwidth,height=\maxheight,keepaspectratio}
% Set default figure placement to htbp
\makeatletter
\def\fps@figure{htbp}
\makeatother
\usepackage{soul}
\setlength{\emergencystretch}{3em} % prevent overfull lines
\providecommand{\tightlist}{%
  \setlength{\itemsep}{0pt}\setlength{\parskip}{0pt}}
\setcounter{secnumdepth}{-\maxdimen} % remove section numbering
\ifLuaTeX
  \usepackage{selnolig}  % disable illegal ligatures
\fi
\IfFileExists{bookmark.sty}{\usepackage{bookmark}}{\usepackage{hyperref}}
\IfFileExists{xurl.sty}{\usepackage{xurl}}{} % add URL line breaks if available
\urlstyle{same}
\hypersetup{
  pdftitle={R Notebook},
  hidelinks,
  pdfcreator={LaTeX via pandoc}}

\title{R Notebook}
\author{}
\date{\vspace{-2.5em}}

\begin{document}
\maketitle

\hypertarget{chip-seq-analysis-of-the-transcription-factor-homeobox-gene-1-from-arabidopsis-thaliana}{%
\section{\texorpdfstring{ChIP-Seq analysis of the transcription factor
HOMEOBOX GENE 1 from \emph{Arabidopsis
thaliana}}{ChIP-Seq analysis of the transcription factor HOMEOBOX GENE 1 from Arabidopsis thaliana}}\label{chip-seq-analysis-of-the-transcription-factor-homeobox-gene-1-from-arabidopsis-thaliana}}

\hypertarget{authors}{%
\paragraph{Authors:}\label{authors}}

Julio Ramírez Guerrero, Julián Román Camacho, Silvestre Ruano Rodríguez,
Manuel Racero de la Rosa, Rafael Rubio Ramos.

\hypertarget{introduction}{%
\subsection{Introduction}\label{introduction}}

\hypertarget{background}{%
\subsubsection{Background}\label{background}}

The motivation for this study lies in the need to understand how genetic
mechanisms and environmental signals are linked to regulate some vital
functions of plants, such as their growth rate.

\hypertarget{objective}{%
\subsubsection{Objective}\label{objective}}

The aim of the study is to identify the regions of the genome where the
transcription factor will bind, in order to obtain information on the
gene regulation of the \emph{ARABIDOPSIS THALIANA HOMEOBOX GENE 1}
(\emph{ATH1}) gene.

\hypertarget{materials-y-methods}{%
\subsection{Materials y Methods}\label{materials-y-methods}}

\hypertarget{experimenta-design}{%
\subsubsection{Experimenta Design}\label{experimenta-design}}

In this task, data deposited in GEO under accession number GSE157332
will be used. The article samples are as follows:

\begin{itemize}
\item
  Chip

  \begin{itemize}
  \tightlist
  \item
    ATH1-GFP\_chipseq\_rep1
  \item
    ATH1-GFP\_chipseq\_rep2
  \item
    ATH1-GFP\_chipseq\_rep3
  \end{itemize}
\item
  Mock

  \begin{itemize}
  \tightlist
  \item
    WT\_control\_Ler\_chipseq\_rep1
  \item
    WT\_control\_Ler\_chipseq\_rep2
  \item
    WT\_control\_Ler\_chipseq\_rep3
  \end{itemize}
\end{itemize}

For Chip samples, an anti-GFP antibody was used on samples of transgenic
\emph{Arabidopsis thaliana} plants expressing ATH1 fused to GFP. In the
Mock samples, the same protocol was followed but using wild-type plants,
i.e. the antibody is used without GFP being present.

\hypertarget{workflow}{%
\subsubsection{Workflow}\label{workflow}}

To identify the target genes of the ATH1 transcription factor, the
following workflow has been followed.

The starting data is a fastq file for each replicate, containing the
sequencing results (after immunoprecipitating the chromatin according to
the experimental design explained). First, we prepare the workspace and
check the quality of the reads from the fastq file.

Next, the reads are mapped to the reference genome, using bowtie2. Using
the result of this alignment, peak calling is performed with macs2. The
aim of this step is to identify sites in the genome where a large number
of reads accumulate in the ChIP samples and not in the control samples.
We will consider these to be regions to which ATH1 binds.

Once the list of sites in the genome is obtained, we try to assign to
each peak found the gene whose activity is regulated, using the Nearest
Downstream Gene method (we look for the nearest downstream gene).
Finally, we use HOMER to search for DNA motifs in the identified peaks.

\begin{figure}
\centering
\includegraphics{images/workflow.webp}
\caption{Workflow for ChIP-Seq data analysis}
\end{figure}

To download the \emph{Arabidopsis thaliana} genome and the corresponding
database annotation, the following scripts were used:

\begin{verbatim}
#!/bin/bash

#SBATCH --job-name=index
#SBATCH --export=ALL
#SBATCH --output=genome_index

## Download reference genome
cd /home/omicas/grupo5/tarea2/genome
wget -O genome.fa.gz https://ftp.ebi.ac.uk/ensemblgenomes/pub/release-58/plants/fasta/arabidopsis_thaliana/dna/Arabidopsis_thaliana.TAIR10.dna.toplevel.fa.gz
gunzip genome.fa.gz

# Download annotation
cd /home/omicas/grupo5/tarea2/annotation
wget -O annotation.gtf.gz https://ftp.ebi.ac.uk/ensemblgenomes/pub/release-58/plants/gtf/arabidopsis_thaliana/Arabidopsis_thaliana.TAIR10.58.gtf.gz
gunzip annotation.gtf.gz

## Construction of the genome index
cd /home/omicas/grupo5/tarea2/genome
bowtie2-build genome.fa index
\end{verbatim}

For the processing of each sample, the following script was used.

\begin{verbatim}
#!/bin/bash

#SBATCH --export=ALL

SAMPLE_DIR=$1
SRR=$2
NAME=$3

## Download fastq files
cd $SAMPLE_DIR
fastq-dump --gzip --split-files $SRR

## Sample quality control and mapping of readings
if [ -f ${SRR}_2.fastq.gz ]
then
   fastqc ${SRR}_1.fastq.gz
   fastqc ${SRR}_2.fastq.gz

   bowtie2 -x ../../genome/index -1 ${SRR}_1.fastq.gz -2 ${SRR}_2.fastq.gz -S $NAME.sam
else
   fastqc ${SRR}_1.fastq.gz

   bowtie2 -x ../../genome/index -U ${SRR}_1.fastq.gz -S $NAME.sam
fi

## The generated sam is transformed into a bam (takes less space)
samtools sort -o $NAME.bam $NAME.sam
## The sam file and the original fastq are deleted
rm $NAME.sam
rm *.fastq.gz
## An index is built for the .bam file
samtools index $NAME.bam
bamCoverage -bs 5 --normalizeUsing CPM --bam $NAME.bam -o $NAME.bw
\end{verbatim}

Finally, the script used for the call to peaks is included.

\begin{verbatim}
#!/bin/bash

#SBATCH --export=ALL

SAMPLE_DIR=$1
IP=$2
CTRL=$3
NAME=$4

cd $SAMPLE_DIR

macs2 callpeak -t $IP -c $CTRL -f BAM --outdir . -n $NAME
\end{verbatim}

In the peak call, IP refers to the ``Chip'' samples, and ``CTRL'' refers
to the ``mock'' samples. Thus, the mock is used to estimate the
background noise. By using three ChIP files and three mock files, this
tool generates a pool with the data from the different replicates before
making the peak call. A peak will be considered to exist when a certain
area accumulates many readings in the ChIP samples but not in the mock
samples. In this step, the default parameters (q-value \textless{} 0.05
and fold-enrichment between 5 and 50) are used.

\hypertarget{results}{%
\subsection{Results}\label{results}}

\begin{Shaded}
\begin{Highlighting}[]
\ControlFlowTok{if}\NormalTok{ (}\SpecialCharTok{!}\FunctionTok{require}\NormalTok{(}\StringTok{"BiocManager"}\NormalTok{, }\AttributeTok{quietly =} \ConstantTok{TRUE}\NormalTok{))}
    \FunctionTok{install.packages}\NormalTok{(}\StringTok{"BiocManager"}\NormalTok{)}

\NormalTok{BiocManager}\SpecialCharTok{::}\FunctionTok{install}\NormalTok{(}\StringTok{"TxDb.Athaliana.BioMart.plantsmart28"}\NormalTok{)}
\NormalTok{BiocManager}\SpecialCharTok{::}\FunctionTok{install}\NormalTok{(}\StringTok{"org.At.tair.db"}\NormalTok{)}
\NormalTok{BiocManager}\SpecialCharTok{::}\FunctionTok{install}\NormalTok{(}\StringTok{"pathview"}\NormalTok{)}
\end{Highlighting}
\end{Shaded}

\hypertarget{analysis-of-the-global-distribution-of-the-cystrome}{%
\subsubsection{Analysis of the global distribution of the
Cystrome}\label{analysis-of-the-global-distribution-of-the-cystrome}}

\begin{Shaded}
\begin{Highlighting}[]
\FunctionTok{library}\NormalTok{(ChIPseeker)}
\FunctionTok{library}\NormalTok{(BiocManager)}
\FunctionTok{library}\NormalTok{(GenomicFeatures)}
\FunctionTok{library}\NormalTok{(TxDb.Athaliana.BioMart.plantsmart28)}
\FunctionTok{library}\NormalTok{(ggupset)}
\NormalTok{txdb }\OtherTok{\textless{}{-}}\NormalTok{ TxDb.Athaliana.BioMart.plantsmart28}
\NormalTok{chip.peaks }\OtherTok{\textless{}{-}} \FunctionTok{readPeakFile}\NormalTok{(}\AttributeTok{peakfile =} \StringTok{"output\_peaks.narrowPeak"}\NormalTok{,}\AttributeTok{header=}\ConstantTok{FALSE}\NormalTok{)}
\FunctionTok{length}\NormalTok{(chip.peaks)}
\end{Highlighting}
\end{Shaded}

\begin{verbatim}
## [1] 783
\end{verbatim}

Peaks represent the regions of the genome where the highest binding of
the protein of interest has been found. In this case we found 783 peaks
where the transcription factor binds.

\begin{Shaded}
\begin{Highlighting}[]
\NormalTok{promoter }\OtherTok{\textless{}{-}} \FunctionTok{getPromoters}\NormalTok{(}\AttributeTok{TxDb=}\NormalTok{txdb, }
                         \AttributeTok{upstream=}\DecValTok{2000}\NormalTok{, }
                         \AttributeTok{downstream=}\DecValTok{2000}\NormalTok{)}
\NormalTok{chip.peakAnno }\OtherTok{\textless{}{-}} \FunctionTok{annotatePeak}\NormalTok{(}\AttributeTok{peak =}\NormalTok{ chip.peaks, }
                             \AttributeTok{tssRegion=}\FunctionTok{c}\NormalTok{(}\SpecialCharTok{{-}}\DecValTok{2000}\NormalTok{, }\DecValTok{2000}\NormalTok{),}
                             \AttributeTok{TxDb=}\NormalTok{txdb)}
\end{Highlighting}
\end{Shaded}

\begin{verbatim}
## >> preparing features information...      2024-07-24 20:09:32 
## >> identifying nearest features...        2024-07-24 20:09:32 
## >> calculating distance from peak to TSS...   2024-07-24 20:09:32 
## >> assigning genomic annotation...        2024-07-24 20:09:32 
## >> assigning chromosome lengths           2024-07-24 20:09:39 
## >> done...                    2024-07-24 20:09:39
\end{verbatim}

\begin{Shaded}
\begin{Highlighting}[]
\FunctionTok{plotAnnoPie}\NormalTok{(chip.peakAnno)}
\end{Highlighting}
\end{Shaded}

\includegraphics{Task_ChIP_seq_files/figure-latex/unnamed-chunk-3-1.pdf}

\begin{Shaded}
\begin{Highlighting}[]
\FunctionTok{plotAnnoBar}\NormalTok{(chip.peakAnno)}
\end{Highlighting}
\end{Shaded}

\includegraphics{Task_ChIP_seq_files/figure-latex/unnamed-chunk-3-2.pdf}

In this series of graphs we will look at how the peaks are distributed
in the different gene structures, to see where the proteins of interest
bind. The first graph is a pie chart showing the distribution of the
peaks, where we can see that they will mainly bind to promoters at a
distance less than or equal to 1kb (50.06\%), then those promoters at a
distance of 1-2 kb (24.27\%). The rest of the elements that are at a
greater distance should not be taken into account, as we assume that it
is not very reliable and there is not much certainty that they regulate
the corresponding gene. The bar graph is more of the same, it shows the
distributions of the distances and as we can see again, most of the
peaks are found in the promoters.

\begin{Shaded}
\begin{Highlighting}[]
\FunctionTok{plotDistToTSS}\NormalTok{(chip.peakAnno,}
              \AttributeTok{title=}\StringTok{"Distribution of genomic loci relative to TSS"}\NormalTok{,}
              \AttributeTok{ylab =} \StringTok{"Genomic Loci (\%) (5\textquotesingle{} {-}\textgreater{} 3\textquotesingle{})"}\NormalTok{)}
\end{Highlighting}
\end{Shaded}

\includegraphics{Task_ChIP_seq_files/figure-latex/unnamed-chunk-4-1.pdf}

This can again be corroborated by the \textbf{distance to TSS} graph,
which shows the distribution of ChIP-Seq peaks with respect to the
position relative to the transcription start site (TSS). About 40\% of
the peaks are upstream of the TSS, between 0-1kb. The same is true for
downstream, but to a lesser extent. There is also a large percentage of
peaks that are 1-3kb away from the TSS.

\begin{Shaded}
\begin{Highlighting}[]
\FunctionTok{upsetplot}\NormalTok{(chip.peakAnno)}
\end{Highlighting}
\end{Shaded}

\includegraphics{Task_ChIP_seq_files/figure-latex/unnamed-chunk-5-1.pdf}
\includegraphics{images/Upsetplot.png}

The \textbf{upsetplot} graph shows where the peaks overlap and their
frequency. It is observed that in most peaks (around 300) the promoters
are located in areas rich in intergenic regions, or for example, the
promoters also overlap with some gene areas, such as exons and 5'UTRs
(around 50).

\begin{Shaded}
\begin{Highlighting}[]
\FunctionTok{plotPeakProf2}\NormalTok{(}\AttributeTok{peak =}\NormalTok{ chip.peaks, }\AttributeTok{upstream =} \FunctionTok{rel}\NormalTok{(}\FloatTok{0.2}\NormalTok{), }\AttributeTok{downstream =} \FunctionTok{rel}\NormalTok{(}\FloatTok{0.2}\NormalTok{),}
              \AttributeTok{conf =} \FloatTok{0.95}\NormalTok{, }\AttributeTok{by =} \StringTok{"gene"}\NormalTok{, }\AttributeTok{type =} \StringTok{"body"}\NormalTok{, }\AttributeTok{nbin =} \DecValTok{800}\NormalTok{,}
              \AttributeTok{TxDb =}\NormalTok{ txdb, }\AttributeTok{weightCol =} \StringTok{"V5"}\NormalTok{,}\AttributeTok{ignore\_strand =}\NormalTok{ F)}
\end{Highlighting}
\end{Shaded}

\begin{verbatim}
## >> binning method is used...2024-07-24 20:09:41
## >> preparing body regions by gene... 2024-07-24 20:09:41
## >> preparing tag matrix by binning...  2024-07-24 20:09:41 
## >> preparing matrix with extension from (TSS-20%)~(TTS+20%)... 2024-07-24 20:09:41
## >> 40 peaks(7.155635%), having lengths smaller than 800bp, are filtered... 2024-07-24 20:09:42
## >> Running bootstrapping for tag matrix...        2024-07-24 20:09:45
\end{verbatim}

\includegraphics{Task_ChIP_seq_files/figure-latex/unnamed-chunk-6-1.pdf}

Finally, in the \textbf{metageneplot}, the peak upstream of the TSS
could indicate a strong binding of the protein to the promoter region of
the gene, which is what is observed in the other plots. We also found
other peaks downstream of the TTS at the end of transcription, perhaps
related to post-transcriptional regulation. And the intermediate zone
between TSS and TTS has no prominent peaks indicating the absence of the
protein in this transcriptional region.

\begin{figure}
\centering
\includegraphics{images/imagePicos1.webp}
\caption{First IGV screenshot.}
\end{figure}

\begin{figure}
\centering
\includegraphics{images/imagePicos2.webp}
\caption{Second IGV screenshot.}
\end{figure}

In the following images we see a series of peaks, which correspond to
our output\_peak. If we look at the sequence we see that there is a high
concentration of reads in the Chip1 sample that does not appear in the
Mock1. These peaks are indicative that our transcription factor (ATH1)
binds to that region. At the bottom of the image we have our genes, the
lines correspond to the introns, the squarer areas are the exons and the
arrow indicates the direction in which it is transcribed.

It is important to mention that this peak is located upstream of the
gene, so it would be the promoter of the gene.

\hypertarget{regulome-analysis-of-the-corresponding-transcription-factor-or-epigenetic-mark.}{%
\subsubsection{Regulome analysis of the corresponding transcription
factor or epigenetic
mark.}\label{regulome-analysis-of-the-corresponding-transcription-factor-or-epigenetic-mark.}}

After the graphical analysis, we saved in a txt file called
``target\_genes.txt'' the genes found in the promoters, which are the
ones we considered important for the distribution of peaks.

\begin{Shaded}
\begin{Highlighting}[]
\NormalTok{chip.annotation }\OtherTok{\textless{}{-}} \FunctionTok{as.data.frame}\NormalTok{(chip.peakAnno)}

\NormalTok{target.genes }\OtherTok{\textless{}{-}} \FunctionTok{c}\NormalTok{(chip.annotation}\SpecialCharTok{$}\NormalTok{geneId[chip.annotation}\SpecialCharTok{$}\NormalTok{annotation }\SpecialCharTok{==} \StringTok{"Promoter (\textless{}=1kb)"}\NormalTok{],}
\NormalTok{                  chip.annotation}\SpecialCharTok{$}\NormalTok{geneId[chip.annotation}\SpecialCharTok{$}\NormalTok{annotation }\SpecialCharTok{==} \StringTok{"Promoter (1{-}2kb)"}\NormalTok{])}
\FunctionTok{length}\NormalTok{(target.genes)}
\end{Highlighting}
\end{Shaded}

\begin{verbatim}
## [1] 582
\end{verbatim}

\begin{Shaded}
\begin{Highlighting}[]
\FunctionTok{write}\NormalTok{(}\AttributeTok{x =}\NormalTok{ target.genes,}\AttributeTok{file =} \StringTok{"target\_genes.txt"}\NormalTok{)}
\end{Highlighting}
\end{Shaded}

Of the 783 peaks found in the IGV file, we will consider those that are
at least 2 kb away from the promoter of a downstream gene. The
percentage of peaks that fulfil this characteristic corresponds to
74.31\% of all peaks found. Thus, we consider a total of 582 genes that
correspond to DNA reads where transcription factors bind.

\href{target_genes.txt}{ATH1 target genes.}

\begin{Shaded}
\begin{Highlighting}[]
\FunctionTok{library}\NormalTok{(clusterProfiler)}
\FunctionTok{library}\NormalTok{(org.At.tair.db)}
\FunctionTok{library}\NormalTok{(enrichplot)}

\NormalTok{chip.enrich.go }\OtherTok{\textless{}{-}} \FunctionTok{enrichGO}\NormalTok{(}\AttributeTok{gene =}\NormalTok{ target.genes,}
                           \AttributeTok{OrgDb         =}\NormalTok{ org.At.tair.db,}
                           \AttributeTok{ont           =} \StringTok{"BP"}\NormalTok{,}
                           \AttributeTok{pAdjustMethod =} \StringTok{"BH"}\NormalTok{,}
                           \AttributeTok{pvalueCutoff  =} \FloatTok{0.05}\NormalTok{,}
                           \AttributeTok{readable      =} \ConstantTok{FALSE}\NormalTok{,}
                           \AttributeTok{keyType =} \StringTok{"TAIR"}\NormalTok{)}




\NormalTok{chip.enrich.kegg }\OtherTok{\textless{}{-}} \FunctionTok{enrichKEGG}\NormalTok{(}\AttributeTok{gene  =}\NormalTok{ target.genes,}
                               \AttributeTok{organism =} \StringTok{"ath"}\NormalTok{,}
                               \AttributeTok{pAdjustMethod =} \StringTok{"BH"}\NormalTok{,}
                               \AttributeTok{pvalueCutoff  =} \FloatTok{0.05}\NormalTok{)}
\end{Highlighting}
\end{Shaded}

\begin{Shaded}
\begin{Highlighting}[]
\FunctionTok{barplot}\NormalTok{(chip.enrich.go,}\AttributeTok{showCategory =} \DecValTok{10}\NormalTok{)}
\end{Highlighting}
\end{Shaded}

\includegraphics{Task_ChIP_seq_files/figure-latex/unnamed-chunk-9-1.pdf}

In the first barplot, the length of the bars indicates the magnitude of
the enrichment for the associated function of each group of genes, so
that the longer the bar length, the stronger the enrichment of the genes
for that process or function. The p-value of each bar indicates whether
the enrichment is significant or not; in this case, red indicates higher
statistical significance (lower p-value).

In this context, when analysing the barplot graph shown, there is high
statistical significance that the selected peaks are transcription
factor binding sites for genes that are involved in cell signalling and
communication processes.

\begin{Shaded}
\begin{Highlighting}[]
\FunctionTok{dotplot}\NormalTok{(chip.enrich.go,}\AttributeTok{showCategory =} \DecValTok{10}\NormalTok{)}
\end{Highlighting}
\end{Shaded}

\includegraphics{Task_ChIP_seq_files/figure-latex/unnamed-chunk-10-1.pdf}

This dotplot indicates the ratio of enriched genes to the total number
of genes associated with a specific function. A high GeneRatio indicates
that a significant proportion of genes associated with a specific
function are present in the set of genes close to the peaks.

In our case, the genes with the highest GeneRatio are those associated
with \textbf{cell signalling and cell communication regulation}
functions, as well as those related to \textbf{regulation of oxygen
levels} as in the previous graph. This set of genes has a low p-value,
indicating that the observation is unlikely to be due to chance. In
summary, the combination of a high GeneRatio and a low p-value
reinforces the evidence that these genes related to cell signalling and
communication are biologically relevant to our gene set, and that this
association is not simply the result of random fluctuations.

\begin{Shaded}
\begin{Highlighting}[]
\FunctionTok{emapplot}\NormalTok{(}\FunctionTok{pairwise\_termsim}\NormalTok{(chip.enrich.go),}\AttributeTok{showCategory =} \DecValTok{10}\NormalTok{,}\AttributeTok{cex\_label\_category=}\FloatTok{0.5}\NormalTok{)}
\end{Highlighting}
\end{Shaded}

\includegraphics{Task_ChIP_seq_files/figure-latex/unnamed-chunk-11-1.pdf}

Thirdly, the emapplot graph is used to visualise similarity
relationships between the different genes under study. The graph shows
relationships between cell regulation and cell communication genes,
forming a triangular figure, and a hexagon is also displayed, indicating
similarity relationships between genes involved in cell response as a
function of oxygen levels in the cell. Finally, there is a set of genes
set apart, represented as an isolated point, related to developmental
functions of the plant after the embryonic stage. Notably, this single
gene contains the highest p-value and the lowest number of genes.

\begin{Shaded}
\begin{Highlighting}[]
\FunctionTok{cnetplot}\NormalTok{(chip.enrich.go,}\AttributeTok{showCategory =} \DecValTok{10}\NormalTok{,}\AttributeTok{cex.params=}\FunctionTok{list}\NormalTok{(}\AttributeTok{gene\_label=}\FloatTok{0.5}\NormalTok{))}
\end{Highlighting}
\end{Shaded}

\includegraphics{Task_ChIP_seq_files/figure-latex/unnamed-chunk-12-1.pdf}

In this Cnetplot graph we can see the grouping of the enriched genes in
3 very well differentiated clusters, which coincide with the previous
graph. The cluster in the upper right-hand zone is related to the
post-embryonic morphogenesis functions of the plant. Meanwhile, the
central cluster shows genes that influence cell regulation and
communication. Finally, the bottom cluster reflects those genes
associated with the regulation of oxygen levels, such as response to
hypoxia, cellular response to oxygen depletion, etc.

\begin{Shaded}
\begin{Highlighting}[]
\FunctionTok{library}\NormalTok{(pathview)}
\NormalTok{df.chip.enrich.kegg }\OtherTok{\textless{}{-}} \FunctionTok{as.data.frame}\NormalTok{(chip.enrich.kegg)}
\FunctionTok{head}\NormalTok{(df.chip.enrich.kegg)}
\end{Highlighting}
\end{Shaded}

\begin{verbatim}
##                                      category         subcategory       ID
## ath04075 Environmental Information Processing Signal transduction ath04075
## ath04016 Environmental Information Processing Signal transduction ath04016
##                                                                     Description
## ath04075 Plant hormone signal transduction - Arabidopsis thaliana (thale cress)
## ath04016    MAPK signaling pathway - plant - Arabidopsis thaliana (thale cress)
##          GeneRatio  BgRatio       pvalue    p.adjust      qvalue
## ath04075     19/89 408/5530 1.847260e-05 0.001071411 0.001030577
## ath04016     10/89 141/5530 7.686507e-05 0.002229087 0.002144131
##                                                                                                                                                                                                 geneID
## ath04075 AT1G17380/AT1G19050/AT1G45249/AT1G72450/AT1G74890/AT2G33860/AT3G03830/AT3G26810/AT3G56800/AT3G63010/AT4G34160/AT4G38850/AT5G53160/AT5G66880/AT1G04250/AT1G10470/AT1G19350/AT2G01570/AT4G00880
## ath04016                                                                                           AT1G05100/AT1G20630/AT2G38470/AT3G56800/AT4G11280/AT5G07180/AT5G53160/AT5G62230/AT5G66880/AT5G47910
##          Count
## ath04075    19
## ath04016    10
\end{verbatim}

\begin{Shaded}
\begin{Highlighting}[]
\FunctionTok{pathview}\NormalTok{(}\AttributeTok{gene.data =}\NormalTok{ target.genes,}
         \AttributeTok{pathway.id =} \StringTok{"ath04075"}\NormalTok{,}
         \AttributeTok{species =} \StringTok{"ath"}\NormalTok{,}\AttributeTok{gene.idtype =} \StringTok{"TAIR"}\NormalTok{)}
\end{Highlighting}
\end{Shaded}

\begin{verbatim}
## [1] "Note: 57 of 551 unique input IDs unmapped."
\end{verbatim}

\begin{figure}
\centering
\includegraphics{ath04075.pathview.png}
\caption{Hormone signalling pathway in Arabidopsis.}
\end{figure}

It is important to note from the image above that the transcription
factor under study regulates those genes coloured red, which is not the
case with uncoloured genes, where it shows no regulation at all.

\hypertarget{enrichment-of-dna-motifs-at-binding-sites}{%
\subsubsection{Enrichment of DNA motifs at binding
sites}\label{enrichment-of-dna-motifs-at-binding-sites}}

Among the known motifs, the most enriched is related to Replumless
(BLH), which is a BELL (plant developmental regulation) transcription
factor related to ATH1.

It can be seen that among the most enriched motifs, the most frequently
repeated motifs belong to the Homeobox family, which is characteristic
of transcription factors involved in the regulation of cell development
and differentiation. Within the motifs associated with the Homeobox (HB)
family, a rather repeated motif, AA\ul{TGATTG,} associated with the ATHB
transcription factors that are more specifically involved in
\textbf{leaf and root development, stress response and growth
regulation,} is clearly visible.

Among the unknown motifs, the second most enriched is \ul{TGATTG}AT,
largely matching the sequence of the most repeated one seen above.
Furthermore, according to the Homer tool it has the best match to the
Homeobox family, suggesting that ATH1 , binds to this motif.

The known motifs with the highest enrichment are shown below.

\begin{longtable}[]{@{}
  >{\centering\arraybackslash}p{(\columnwidth - 6\tabcolsep) * \real{0.2361}}
  >{\centering\arraybackslash}p{(\columnwidth - 6\tabcolsep) * \real{0.2361}}
  >{\centering\arraybackslash}p{(\columnwidth - 6\tabcolsep) * \real{0.2917}}
  >{\centering\arraybackslash}p{(\columnwidth - 6\tabcolsep) * \real{0.2361}}@{}}
\toprule\noalign{}
\begin{minipage}[b]{\linewidth}\centering
Clasifcation
\end{minipage} & \begin{minipage}[b]{\linewidth}\centering
Known Motifs
\end{minipage} & \begin{minipage}[b]{\linewidth}\centering
Name
\end{minipage} & \begin{minipage}[b]{\linewidth}\centering
P-value
\end{minipage} \\
\midrule\noalign{}
\endhead
\bottomrule\noalign{}
\endlastfoot
1 & -GTC--ATCA & Replumless(BLH)/Arabidopsis-RPL.GFP & 1e-88 \\
2 & GA-G-GAC-GG- & Knotted(Homeobox)/Corn-KN1 & 1e-71 \\
3 & TGACGTCAC- & FEA4(bZIP)/Corn-FEA4 & 1e-23 \\
4 & CA-TCATTCA & WUS1(Homeobox)/colamp-WUS1 & 1e-21 \\
5 & AA\ul{TGATTG} & ATHB5(HB)/colamp-ATHB5 & 1e-18 \\
6 & AA\ul{TGATTG} & ATHB7(Homeobox)/col-ATHB7 & 1e-17 \\
13 & AA\ul{TGATTG} & ATHB6(Homeobox)/col-ATHB6 & 1e-8 \\
\end{longtable}

En la tabla se muestran los motivos desconocidos más enriquecidos.

\begin{longtable}[]{@{}
  >{\centering\arraybackslash}p{(\columnwidth - 6\tabcolsep) * \real{0.2394}}
  >{\centering\arraybackslash}p{(\columnwidth - 6\tabcolsep) * \real{0.2394}}
  >{\centering\arraybackslash}p{(\columnwidth - 6\tabcolsep) * \real{0.2817}}
  >{\centering\arraybackslash}p{(\columnwidth - 6\tabcolsep) * \real{0.2394}}@{}}
\toprule\noalign{}
\begin{minipage}[b]{\linewidth}\centering
Clasification
\end{minipage} & \begin{minipage}[b]{\linewidth}\centering
Unknown Motifs
\end{minipage} & \begin{minipage}[b]{\linewidth}\centering
Best Match
\end{minipage} & \begin{minipage}[b]{\linewidth}\centering
P-value
\end{minipage} \\
\midrule\noalign{}
\endhead
\bottomrule\noalign{}
\endlastfoot
1 & GTCACATC & Knotted(Homeobox)/Corn-KN1 & 1e-73 \\
2 & \ul{TGATTG}AT & WUS1(Homeobox)/colamp-WUS1 & 1e-53 \\
3 & -ACCTGTC & bZIP18(bZIP)/colamp-bZIP18 & 1e-29 \\
4 (posible falso positivo) & GTCTTTTC &
Replumless(BLH)/Arabidopsis-RPL.GFP & 1e-11 \\
\end{longtable}

\hypertarget{conclusions}{%
\subsection{Conclusions}\label{conclusions}}

The most relevant conclusions of our study are the following:

\begin{itemize}
\item
  Transcription factors of the Homeobox family (including ATH1)
  repeatedly bind to a well-known TGATTG motif, as shown by Homer
  analysis.
\item
  The ATH1 transcription factor regulates a multitude of vital plant
  processes, such as regulation of cell signalling and cell
  communication, regulation of oxygen levels and leaf and root
  development, post-embryonic morphogenesis, among others.
\end{itemize}

Below are the common conclusions drawn from our study and the article:

\begin{itemize}
\item
  Both the findings in the article and our results agree that ATH1
  regulates genes involved in cell signalling and communication.
\item
  ATH1 regulates the DELLA protein (which is regulated by gibberellins),
  which is closely related to plant morphology and its regulation.
\item
  The most repeated motif among the enriched motifs, both in our study
  and in the paper, is TGATTG , which is in agreement with the results
  extracted from the analysis with the Homer software.
\end{itemize}

A possible experiment to validate our results would be to compare a
control plant and an ath1- mutant to check the type of plant growth in
response to hormones, such as auxins. The control plants without hormone
treatment grow following a rosette-like structure. However, the ath1-
mutant plants grow following a default caulescent morphology. To our
astonishment, after several days in which the plant structure was
already established, auxins were inoculated to our mutant, resulting in
a morphological change that led to a shortening of the plant internodes,
resulting in a rosette shape.

\hypertarget{references}{%
\subsection{References}\label{references}}

Bencivenga, S. et al.~(2021). ARABIDOPSIS THALIANA HOMEOBOX GENE 1
controls plant architecture by locally restricting environmental
responses. \emph{Proc Natl Acad Sci U S A.118(17)}. PMID: 33888582.

\end{document}
